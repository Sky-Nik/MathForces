Единственным свидетелем по делу о налёте на продуктовый магазин \guillemotleftОдуванчик\guillemotright оказалась пенсионерка Клавдия Петровна. В ходе предварительного следствия выяснилось, что в налёте участвовали ровно $k$ преступников. Пенсионерка утверждает, что точно запомнила всех $k$ налётчиков. На процедуру опознания привели $9$ подозреваемых. Следователь может задавать любые вопросы, предполагающие ответы \guillemotleftда\guillemotright или \guillemotleftнет\guillemotright. За какое наименьшее число вопросов следователь может определить всех преступников, если (a) $k = 2$; (b*) $k = 5$?