На столе в ряд лежат $36$ различных карт. Над ними производят магическую операцию, в результате которой первая карта оказывается на месте под номером $n_1$, вторая --- на месте под номером $n_2$, и так вплоть до $36$-й, которая оказывается на месте $n_{36}$ ($n_1, \ldots, n_{36}$ --- попарно различные числа от $1$ до $36$). Когда эту операцию сделали $5$ раз, все карты оказались в исходном положении. (a) Верно ли, что существует карта, всё время остававшаяся на своём месте? Сколько таких карт могло быть? (b) Тот же вопрос, если $5$ заменить на $25$. (c) А если добавить условие, что на $25$-й раз карты впервые окажутся в исходном положении?