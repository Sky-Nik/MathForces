\documentclass[a4paper, 12pt]{article}
\usepackage[T1, T2A]{fontenc}
\usepackage[utf8]{inputenc}
\usepackage[english, ukrainian]{babel}

\usepackage{amsmath, amssymb}
\usepackage{multicol, graphicx, float}

\usepackage[margin=1cm]{geometry}

\allowdisplaybreaks
\setlength\parindent{0pt}

\newcommand{\NN}{\mathbb{N}}
\newcommand{\ZZ}{\mathbb{Z}}
\newcommand{\QQ}{\mathbb{Q}}
\newcommand{\RR}{\mathbb{R}}
\newcommand{\CC}{\mathbb{C}}

\usepackage{amsthm}
\theoremstyle{definition}
\newtheorem{problem}{Задача}

\begin{document}

\title{Тренировка №9: Информация и кодирование. Добавка}
\maketitle

\begin{problem}[Язык племени АУ]
	В языке племени АУ две буквы --- \guillemotleft a\guillemotright и \guillemotleft y\guillemotright. Некоторые последовательности этих букв являются словами, причём в каждом слове не меньше одной и не больше $13$ букв. Известно, что если написать подряд любые два слова, то полученная последовательность букв не будет словом. Найдите максимальное возможное количество слов в таком языке.
\end{problem}

\begin{problem}[Гоблинские банды]
	В пещере живут $80$ гоблинов. Кроме этого в пещере действуют несколько банд, каждая банда состоит ровно из $3$ гоблинов. Известно, что в любых двух бандах не более, чем один общий гоблин. Докажите, что можно выбрать $13$ гоблинов так, что среди них не будет ни одной полноценной банды.
\end{problem}

\begin{problem}[Попугай-обманщик и сокровища]
	Перед вами три сундука. В некоторых из них спрятаны сокровища, но вы не знаете в каких именно (сокровища могут находится как во всех сундуках, так и нигде). Попугай, который сидит рядом с сундуками, знает где находятся сокровища. Он согласился ответить на наши любые шесть вопросов, но при этом он может не более одного раза соврать (он умеет отвечать только \guillemotleft да\guillemotright или \guillemotleft нет\guillemotright). Как выяснить, где находятся сокровища?
\end{problem}

\end{document}