\documentclass[a4paper, 12pt]{article}
\usepackage[T1, T2A]{fontenc}
\usepackage[utf8]{inputenc}
\usepackage[english, ukrainian]{babel}

\usepackage{amsmath, amssymb}
\usepackage{multicol, graphicx, float}

\usepackage[margin=1cm]{geometry}

\allowdisplaybreaks
\setlength\parindent{0pt}

\newcommand{\NN}{\mathbb{N}}
\newcommand{\ZZ}{\mathbb{Z}}
\newcommand{\QQ}{\mathbb{Q}}
\newcommand{\RR}{\mathbb{R}}
\newcommand{\CC}{\mathbb{C}}

\usepackage{amsthm}
\theoremstyle{definition}
\newtheorem{problem}{Задача}

\begin{document}

\title{Тренировка №3: Считаем ребра в графах}
\maketitle

\begin{problem}[Егор и Леша]
	На занятие кружка пришли $24$ школьника, среди них отличник Егор. Руководитель спросил у каждого из них, кроме Леши, сколько у них знакомых среди остальных пришедших. В ответ прозвучали только числа $3$ и $5$. Докажите, что Леша с кем-нибудь знаком.
\end{problem}

\begin{problem}[только Т-образные перекрестки]
	В некотором городе на каждом перекрёстке сходятся ровно три улицы. Улицы раскрашены в три цвета так, что на каждом перекрёстке сходятся улицы трёх разных цветов. Из города выходят три дороги. Докажите, что они имеют разные цвета.
\end{problem}

\begin{problem}[Кавелерия дивизия №777]
	Можно ли расставить $777$ шахматных коней на доске $2018 \times 2018$ так, чтобы каждый из них бил ровно $4$ других?
\end{problem}

\begin{problem}[Заходят эльф, гном, и хоббит в бар&hellip;]
	В компании пять эльфов, пять гномов и один хоббит. У каждого эльфа по семь знакомых в этой компании, а у каждого гнома по два. Сколько знакомых в этой компании у хоббита?
\end{problem}

\begin{problem}[Либо трое попарно знакомых, либо&hellip;]
	Докажите, что среди любых шести человек есть либо трое попарно знакомых, либо трое попарно незнакомых.
\end{problem}

\begin{problem}[Либо четверо попарно знакомых, либо&hellip;]
	Докажите, что среди любых девяти человек есть либо четверо попарно знакомых, либо трое попарно незнакомых.
\end{problem}

\begin{problem}[В кругу друзей]
	Есть семь человек, любых шестерых можно посадить за круглый стол так, чтобы каждый сидел рядом с двумя своими друзьями. Докажите, что можно так посадить всех семерых, что каждый будет сидеть рядом со своими друзьями.
\end{problem}

\begin{problem}[Скажи &laquo;нет&raquo; треугольникам!]
	В графе $100$ вершин, нет треугольников, степень каждой вершины больше $40$. Доказать, что в этом графе нет циклов длины $5$.
\end{problem}

\begin{problem}[Максимум ребер без клик]
	Какое наибольшее число рёбер может быть в $90$-вершинном графе, в котором (a) нет треугольников; (b) нет полного подграфа из четырёх вершин?
\end{problem}

\begin{problem}[Ошибка статистика]
	В стране $2000$ городов. Каждый город связан беспосадочными двусторонними авиалиниями с некоторыми другими городами, причём для каждого города число исходящих из него авиалиний есть степень двойки (то есть $1, 2, 4, \ldots$). Для каждого города A статистик подсчитал количество маршрутов, имеющих не более одной пересадки, связывающих A с другими городами, а затем просуммировал полученные результаты по всем $2000$ городам. У него получилось $100000$. Докажите, что статистик ошибся.
\end{problem}

\end{document}