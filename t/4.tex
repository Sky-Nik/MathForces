\documentclass[a4paper, 12pt]{article}
\usepackage[T1, T2A]{fontenc}
\usepackage[utf8]{inputenc}
\usepackage[english, ukrainian]{babel}

\usepackage{amsmath, amssymb}
\usepackage{multicol, graphicx, float}

\usepackage[margin=1cm]{geometry}

\allowdisplaybreaks
\setlength\parindent{0pt}

\newcommand{\NN}{\mathbb{N}}
\newcommand{\ZZ}{\mathbb{Z}}
\newcommand{\QQ}{\mathbb{Q}}
\newcommand{\RR}{\mathbb{R}}
\newcommand{\CC}{\mathbb{C}}

\usepackage{amsthm}
\theoremstyle{definition}
\newtheorem{problem}{Задача}

\begin{document}

\title{Тренировка №4: Считаем ребра в графах}
\maketitle

\begin{problem}[Чётное число общих знакомых]
	Докажите, что среди $100$ человек найдутся двое, у которых чётное число общих знакомых.
\end{problem}

\begin{problem}[Юный паяльщик]
	Каждая деталь конструктора \guillemotleftЮный паяльщик\guillemotright --- это скобка в виде буквы П, состоящая из трёх единичных отрезков. Можно ли из деталей этого конструктора спаять полный проволочный каркас куба $2 \times 2 \times 2$, разбитого на кубики $1 \times 1 \times 1$? (Каркас состоит из 27 точек, соединённых единичными отрезками; любые две соседние точки должны быть соединены ровно одним проволочным отрезком.)
\end{problem}

\begin{problem}[Heil Hidra!]
	Гидры состоят из голов и шей (каждая шея соединяет ровно две головы). Одним ударом меча можно снести все шеи, выходящие из какой-то головы A гидры. Но при этом из головы A мгновенно вырастает по одной шее во все головы, с которыми A не была соединена. Геракл побеждает гидру, если ему удастся разрубить её на две несвязанные шеями части. Найдите наименьшее $N$, при котором Геракл сможет победить любую стошеюю гидру, нанеся не более чем $N$ ударов.
\end{problem}

\end{document}