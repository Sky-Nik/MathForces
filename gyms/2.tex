\documentclass[a4paper, 12pt]{article}
\usepackage[T1, T2A]{fontenc}
\usepackage[utf8]{inputenc}
\usepackage[english, ukrainian]{babel}

\usepackage{amsmath, amssymb}
\usepackage{multicol, graphicx, float}

\usepackage[margin=1cm]{geometry}

\allowdisplaybreaks
\setlength\parindent{0pt}

\newcommand{\NN}{\mathbb{N}}
\newcommand{\ZZ}{\mathbb{Z}}
\newcommand{\QQ}{\mathbb{Q}}
\newcommand{\RR}{\mathbb{R}}
\newcommand{\CC}{\mathbb{C}}

\usepackage{amsthm}
\theoremstyle{definition}
\newtheorem{problem}{Задача}

\begin{document}

\title{Тренировка №2: Клеточная оценка+пример}
\maketitle

\begin{problem}[Мирная шахматная сходка]
	Какое наибольшее количество (a) ладей; (b) слонов; (c) королей; (d) коней можно расставить на шахматной доске так, что никакие две фигуры не были друг друга?
\end{problem}

\begin{problem}[Почти кубик Рубика 2&times;2&times;2]
	Каждую грань кубика разбили на четыре одинаковых квадрата, а затем раскрасили эти квадраты в несколько цветов так, что квадраты, имеющие общую сторону, оказались окрашенными в разные цвета. Какое наибольшее количество квадратов одного цвета могло получится?
\end{problem}

\begin{problem}[Гепард гепарду друг]
	Дана клетчатая доска $1000 \times 1000$. Фигура гепард из произвольной клетки $x$ бьёт все клетки квадрата $19 \times 19$ с центральной клеткой $x$, за исключением клеток, находящихся с $x$ в одном столбце или одной строке. Какое наибольшее количество гепардов, не бьющих друг друга, можно расставить на доске?
\end{problem}

\begin{problem}[Ладьи бьют друг друга]
	Какое наибольшее количество ладей можно расставить на шахматной доске, чтобы каждая била ровно две другие?
\end{problem}

\begin{problem}[Короли бьют всю доску]
	Какое наименьшее число королей необходимо поставить на шахматную доску так, чтобы любая клетка была побита хотя бы одним королем (король бьет клетку, на которой стоит сам)?
\end{problem}

\begin{problem}[Карлсон, Малыш, и квадратный торт]
	Каждый день Фрекен Бок испекает квадратный торт размером $3 \times 3$. Карлсон немедленно вырезает себе из него четыре квадратных кусков размером $1 \times 1$ со сторонами, параллельными сторонам торта (не обязательно по линиям сетки $3 \times 3$). После этого Малыш вырезает себе из оставшейся части торта квадратный кусок со сторонами, также параллельными сторонам торта. На какой наибольший кусок торта может рассчитывать Малыш вне зависимости от действий Карлсона?
\end{problem}

\begin{problem}[&laquo;Равномерное&raquo; распределение фишек]
	Какое (a) наибольшее; (b) наименьшее количество фишек можно поставить на шахматную доску так, чтобы в каждом квадрате $3 \times 3$ стояло ровно по $3$ фишки?
\end{problem}

\begin{problem}[Черно-белые треугольники]
	В каждой клетке квадрата $n \times n$ провели две диагонали, в результате чего квадрат оказался разбит на $4n^2$ треугольников. Все треугольники покрасили в черный или белый цвет, так что при этом каждый белый треугольник имеет общую сторону хотя бы с одним черным. Какое наименьшее количество черных треугольников могло быть в такой раскраске?
\end{problem}

\begin{problem}[Петя, Вася, и дорогая игра]
	Петя и Вася играют в следующую игру. Вася заполняет числами от $1$ до $100$ клетки таблицы $10 \times 10$ (каждое --- по одному разу). Петя хочет пройти шахматным королём от левого края доски до правого. При этом если он ставит короля на какую-то клетку, то он обязан заплатить Васе такое число рублей, которое на ней написано. Сколько Петя заплатит Васе при правильной игре? (Петя хочет заплатить как можно меньше, Вася --- получить как можно больше.)
\end{problem}

\end{document}