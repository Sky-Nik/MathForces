
\input{gym.sty}
\begin{document}
\title{Тренування №1: Разные раскраски}
\maketitle
\begin{problem}[8&times;8 на 1&times;3]
	Можно ли разбить квадрат $8 \times 8$ с отрезанным уголком на прямоугольники $1 \times 3$?

\end{problem}

\begin{problem}[10&times;10 на 1&times;4]
	Можно ли разрезать квадрат $10 \times 10$ на прямоугольники $1 \times 4$?
\end{problem}

\begin{problem}[10&times;10 на Т-тетраминошки]
	Можно ли разрезать квадрат $10 \times 10$ на Т-тетраминошки? (Домино, тримино,
тетрамино, пентамино, \dots --- фигурки, составленные из 2, 3, 4, 5, \dots клеток
соответственно.)
\end{problem}

\begin{problem}[8&times;8 и странные доминошки]
	Можно ли разрезать квадрат $8 \times 8$ на $17$ вертикальных и $15$ горизонтальных доминошек?
\end{problem}

\begin{problem}[Куб 3&times;3&times;3 без трех уголков]
	Можно ли куб $3 \times 3 \times 3$, у которого вырезаны три угловых кубика, разбить на параллелепипеды $1 \times 1 \times 2$?
\end{problem}

\begin{problem}[Куб 5&times;5&times;5 без противоположных уголков]
	Можно ли куб $5 \times 5 \times 5$, у которого вырезаны два противоположных угловых кубика, разбить на параллелепипеды $1 \times 1 \times 3$?
\end{problem}

\begin{problem}[Перекрась квадрат!]
	Квадрат $8 \times 8$ клеток выкрашен в белый цвет. Разрешается выбрать в нём любой прямоугольник из трёх клеток и перекрасить их все в противоположный цвет (белые в чёрный, чёрные --- в белый). Удастся ли несколькими такими операциями перекрасить весь квадрат в чёрный цвет?
\end{problem}

\begin{problem}[Доску 8&times;8 таки разрезали]
	Из доски $8 \times 8$ вырезали одну клетку так, что остаток можно разрезать на прямоугольники $3 \times 1$. Укажите все клетки, которые могут быть вырезаны и докажите, что других нет.
\end{problem}

\begin{problem}[&laquo;Ну Барсик, ну ещё капельку&raquo;]
	Из листа клетчатой бумаги размером $29 \times 29$ клеточек вырезали $99$ квадратиков $2 \times 2$ (режут по линиям). Доказать, что из оставшейся части листа можно вырезать ещё хотя бы один такой же квадратик.
\end{problem}

\begin{problem}[Хорошие клетки в 11&times;11]
	Клетку квадрата $11 \times 11$ назовем хорошей, если после её удаления оставшуюся часть можно разрезать на прямоугольники $1 \times 4$. Сколько существует хороших клеток?
\end{problem}

\begin{problem}[Сотовая игра]
	Петя и Вася играют в игру на решётке в виде сот. Петя закрашивает в чёрный цвет два соседних шестиугольника, а Вася стирает один любой закрашенный. Какое максимальное количество закрашенных шестиугольников подряд может гарантировано получить Петя?
\end{problem}

\end{document}