
\documentclass[a4paper, 12pt]{article}
\usepackage[T1, T2A]{fontenc}
\usepackage[utf8]{inputenc}
\usepackage[english, ukrainian]{babel}

\usepackage{amsmath, amssymb}
\usepackage{multicol, graphicx, float}

\usepackage[margin=1cm]{geometry}

\allowdisplaybreaks
\setlength\parindent{0pt}

\newcommand{\NN}{\mathbb{N}}
\newcommand{\ZZ}{\mathbb{Z}}
\newcommand{\QQ}{\mathbb{Q}}
\newcommand{\RR}{\mathbb{R}}
\newcommand{\CC}{\mathbb{C}}

\usepackage{amsthm}
\theoremstyle{definition}
\newtheorem{problem}{Задача}
\begin{document}
\title{Тренування №8: Информация и кодирование}
\maketitle
\begin{problem}[Легкая фальшивка]
	Среди $100$ монет одна фальшивая. За какое минимальное число взвешиваний на двухчашечных весах можно найти фальшивую монету, если известно, что она легче настоящих?
\end{problem}

\begin{problem}[Магазин &laquo;Одуванчик&raquo; и Клавдия Петровна]
	Единственным свидетелем по делу о налёте на продуктовый магазин \guillemotleftОдуванчик\guillemotright оказалась пенсионерка Клавдия Петровна. В ходе предварительного следствия выяснилось, что в налёте участвовали ровно $k$ преступников. Пенсионерка утверждает, что точно запомнила всех $k$ налётчиков. На процедуру опознания привели $9$ подозреваемых. Следователь может задавать любые вопросы, предполагающие ответы \guillemotleftда\guillemotright или \guillemotleftнет\guillemotright. За какое наименьшее число вопросов следователь может определить всех преступников, если (a) $k = 2$; (b*) $k = 5$?
\end{problem}

\begin{problem}[Золото, серебро, подделки]
	Имеется $n$ монет двух цветов, среди которых ровно одна фальшивая. Все настоящие монеты весят одинаково вне зависимости от цвета. Если серебристая монета фальшивая, то она легче настоящей. Если золотистая монета фальшивая --- то тяжелее. При каком максимальном $n$ можно найти фальшивую монету с помощью двухчашечных весов и $k$ взвешиваний, если (a) $k = 2$; (b*) $k$ --- любое?
\end{problem}

\begin{problem}[Две группы]
	В классе $16$ учеников. Каждый месяц учитель делит класс на две группы. Какое наименьшее количество месяцев должно пройти, чтобы каждые два ученика в какой-то из месяцев оказались в разных группах?
\end{problem}

\begin{problem}[11 мудрецов и тысяча цветов]
	Одиннадцати мудрецам завязывают глаза и надевают каждому на голову колпак одного из $1000$ цветов. После этого им глаза развязывают, и каждый видит все колпаки, кроме своего. Затем одновременно каждый показывает остальным одну из двух карточек --- белую или чёрную. После  этого все должны одновременно назвать цвет своих колпаков. Удастся ли это?
\end{problem}

\begin{problem}[Энакин и имена]
	За столом сидят $2018$ джедаев. Любознательный Энакин хочет узнать, как их зовут (у всех джедаев разные имена). Он может показать на несколько джедаев пальцем и попросить магистра Йоду перечислить все их имена. К сожалению, порядок, в котором Йода перечисляет имена, может быть произвольным. Какое наименьшее количество раз Энакину придется отвлечь магистра Йоду от медитации?
\end{problem}

\begin{problem}[Лабораторные крысы и яд]
	У неаккуратного лаборанта перепутались пометки пробирок, и среди его пробирок с препаратами есть одна с ядом. У лаборанта есть $4$ подопытные крысы. Каждый день в 10:00 каждой крысе можно сделать инъекцию со смесью содержимого нескольких пробирок. Если среди них была пробирка с ядом, крыса погибнет ровно в 17:00, иначе крыса выживет. Для какого наибольшего количества пробирок за $4$ дня гарантированно можно выяснить, в какой пробирке яд?
\end{problem}

\end{document}