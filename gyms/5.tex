\documentclass[a4paper, 12pt]{article}
\usepackage[T1, T2A]{fontenc}
\usepackage[utf8]{inputenc}
\usepackage[english, ukrainian]{babel}

\usepackage{amsmath, amssymb}
\usepackage{multicol, graphicx, float}

\usepackage[margin=1cm]{geometry}

\allowdisplaybreaks
\setlength\parindent{0pt}

\newcommand{\NN}{\mathbb{N}}
\newcommand{\ZZ}{\mathbb{Z}}
\newcommand{\QQ}{\mathbb{Q}}
\newcommand{\RR}{\mathbb{R}}
\newcommand{\CC}{\mathbb{C}}

\usepackage{amsthm}
\theoremstyle{definition}
\newtheorem{problem}{Задача}

\begin{document}

\title{Тренировка №5: Делимость и остатки}
\maketitle

\begin{problem}[Раздели камни]
	Камни лежат в трёх кучках: в одной --- $51$ камень, в другой --- $49$ камней, а в третьей --- $5$ камней. Разрешается объединять любые кучки в одну, а также разделять кучку из чётного количества камней на две равные. Можно ли получить $105$ кучек по одному камню в каждой?
\end{problem}

\begin{problem}[Разбойники и мешок монет]
	Шайка разбойников отобрала у купца мешок монет. Каждая монета стоит целое число грошей. Оказалось, что какую бы монету ни отложить, оставшиеся монеты можно разделить между разбойниками так, чтобы каждый получил одинаковую сумму в грошах. Докажите, что если отложить одну монету, то число монет разделится на число разбойников.
\end{problem}

\begin{problem}[Выдели число]
	Дана бесконечная вправо последовательность цифр и натуральное число $l$. Докажите, что можно выбрать несколько цифр подряд, образующих число, делящееся на $l$, если (a) $l = 9$; (b) $l$ --- нечетное число, не делящееся на $5$.
\end{problem}

\begin{problem}[Незнайка на Луне и сдача]
	На Луне имеют хождение монеты достоинством в $1$, $15$ и $50$ фертингов. Незнайка отдал за покупку несколько монет и получил сдачу --- на одну монету больше. Какова наименьшая возможная цена покупки?
\end{problem}

\begin{problem}[Бесконечную ленту разрезали&hellip;]
	Все натуральные числа выписали подряд без промежутков на бесконечную ленту: $123456789101112\ldots$. Затем ленту разрезали на полоски по $7$ цифр в каждой. Докажите, что любое семизначное число (a) встретится хотя бы на одной из полосок; (b) встретится на бесконечном числе полосок.
\end{problem}

\begin{problem}[Магическая перетасовка]
	На столе в ряд лежат $36$ различных карт. Над ними производят магическую операцию, в результате которой первая карта оказывается на месте под номером $n_1$, вторая --- на месте под номером $n_2$, и так вплоть до $36$-й, которая оказывается на месте $n_{36}$ ($n_1, \ldots, n_{36}$ --- попарно различные числа от $1$ до $36$). Когда эту операцию сделали $5$ раз, все карты оказались в исходном положении. (a) Верно ли, что существует карта, всё время остававшаяся на своём месте? Сколько таких карт могло быть? (b) Тот же вопрос, если $5$ заменить на $25$. (c) А если добавить условие, что на $25$-й раз карты впервые окажутся в исходном положении?
\end{problem}

\end{document}