\documentclass[a4paper, 12pt]{article}
\usepackage[T1, T2A]{fontenc}
\usepackage[utf8]{inputenc}
\usepackage[english, ukrainian]{babel}

\usepackage{amsmath, amssymb}
\usepackage{multicol, graphicx, float}

\usepackage[margin=1cm]{geometry}

\allowdisplaybreaks
\setlength\parindent{0pt}

\newcommand{\NN}{\mathbb{N}}
\newcommand{\ZZ}{\mathbb{Z}}
\newcommand{\QQ}{\mathbb{Q}}
\newcommand{\RR}{\mathbb{R}}
\newcommand{\CC}{\mathbb{C}}

\usepackage{amsthm}
\theoremstyle{definition}
\newtheorem{problem}{Задача}

\begin{document}

\title{Тренировка №10: Деревья}
\maketitle

\begin{problem}[Остов есть? А если найду?]
	Для любого связного графа существует остовное дерево.
\end{problem}

\begin{problem}[Остов остову рознь]
	Существует ли граф, у которого есть два остовных дерева без общих ребер?
\end{problem}

\begin{problem}[Удали вершину]
	Докажите, что в любом связном графе можно удалить вершину вместе со всеми выходящими из нее рёбрами так, чтобы он остался связным.
\end{problem}

\begin{problem}[Браконьер и инспектор полиции]
	Невод браконьера представляет собой прямоугольную сетку $100 \times 100$ клеток. После каждой поимки инспектор рыбоохраны обрезает в неводе одну веревочку (указанную браконьером), так, чтобы невод не распался на части. Сколько задержаний может выдержать браконьер до разрушения своего инструмента?
\end{problem}

\begin{problem}[Вассалы, сюзерены, и бароны]
	В некотором королевстве было $32$ рыцаря. Некоторые из них были вассалами других (вассал может иметь только одного сюзерена, причём сюзерен всегда богаче своего вассала). Рыцарь, имевший не менее четырёх вассалов, носил титул барона. Какое наибольшее число баронов могло быть при этих условиях? (В королевстве действовал закон: \guillemotleft вассал моего вассала --- не мой вассал\guillemotright.)
\end{problem}

\begin{problem}[Три авиакомпании]
	В стране $15$ городов, некоторые из них соединены авиалиниями, принадлежащими трём авиакомпаниям. Известно, что даже если любая из авиакомпаний прекратит полеты, можно будет добраться из каждого города в любой другой (возможно, с пересадками), пользуясь рейсами оставшихся двух компаний. Какое наименьшее количество авиалиний может быть в стране?
\end{problem}

\begin{problem}[Экономим на циклах]
	В стране $50$ городов и $100$ дорог между ними. Из любого города можно добраться до другого. Докажите, что можно закрыть все дороги некоторого циклического маршрута так, чтобы по-прежнему из любого города можно было добраться до любого другого.
\end{problem}

\begin{problem}[Открой 50, закрой одну]
	В стране несколько городов, некоторые пары городов соединены дорогами, причём между каждыми двумя городами существует единственный несамопересекающийся путь по дорогам. Известно, что в стране ровно $100$ городов, из которых выходит по одной дороге. Докажите, что можно построить $50$ новых дорог так, что после этого даже при закрытии любой дороги можно будет из каждого города попасть в любой другой.
\end{problem}

\begin{problem}[Прокати кубик]
	Раскрашенный в чёрный и белый цвета кубик с гранью в одну клетку поставили на одну из клеток шахматной доски и прокатили по ней так, что кубик побывал на каждой клетке ровно по одному разу. Можно ли так раскрасить кубик и так прокатить его по доске, чтобы каждый раз цвета клетки и соприкоснувшейся с ней грани совпадали?
\end{problem}

\end{document}